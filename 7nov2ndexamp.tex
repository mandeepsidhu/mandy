\documentclass{article}
\title{In Class Activity}
\author{Mandeep Kaur}
\date{\today}
\begin{document}
\maketitle
\section*{Question1}
a)   $ \{0 ,2 ,4,6,8 \}$   					\\ 
Ans :  $P \backslash A_{9}$ 					\\
b)   	$ \{11,13,15,17, .... ,101 \}$ 				\\
Ans :   $ I_{11} \cap A_{101}$					\\
c)  $\emptyset$ 					\\
Ans :   		$ P \cap I$			\\
d)   	$ \{1,10\}$ 				\\
Ans :   	$ (B_{1} \cap A_{10}) - (B_{2} \cap A_{9})$					\\
e)   		$ \{n , n+1 ,..... m \},$  where $n<=m$			\\
Ans :   				\\



\section*{Question2}


a) $\forall n \exists m ( n^2 < m)$ \\
Ans : True because for all n we can choose  $ n^2 <m . $ \\
\\ 
b) $\exists n \forall m (n <m ^2) $ \\ 
Ans : True because for all m we can choose  $ n < m^2  .$ \\
\\  
c) $\forall n \exists m  (n+ m =0) $\\ 
Ans : True because for all n we can choose value of m and make this predicate true.\\
\\
d) $\exists n \forall m (nm=m)  $\\  
Ans : True  by putting n = 1,this predicate is always true and make it true we need only one  magic value of n for all values of m . \\
\\
e) $\exists n \exists m (n^2 + m^2 = 5) $  \\ 
Ans : True because   n=1 and   m =2 we can get our solution and here we need only one set which makes it true.\\
\\
f) $\exists n \exists m  (n^2 + m^2 =6) $ \\
Ans : False because we do not get any solution for it not even a single set which makes it true.\\
\\
g) $\exists n \exists m (n+m =4 \wedge n-m =1 )  $ \\
Ans : False because this predicate do not have any solution not even a single. \\
\\
h) $\exists n \exists m   (n+m =4 \wedge n-m =2 )$ \\
Ans : True . Here we need only one solution to make it true and one solution is n=3 , m=1.\\
\\
i) $\forall n \forall m \exists p   (p=(m+n)/2) $\\
Ans : False  . counter example is  that when  sum of m and n is not divisible by 2 . example m=2 and n=5. \\
\\

\end{document}